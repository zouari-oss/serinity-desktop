\documentclass[a4paper,12pt]{article}
\usepackage[utf8]{inputenc}
\usepackage{graphicx}
\usepackage{geometry}
\usepackage{array}
\usepackage{longtable}

\geometry{left=2.5cm,right=2.5cm,top=2.5cm,bottom=2.5cm}

\begin{document}

%---------------------------------------
% Title page
%---------------------------------------
\begin{titlepage}
    \begin{center}
        % Logo
        \includegraphics[width=0.5\textwidth]{../img/logo/serinity-logo.png}\\[1cm]
        \textsc{\Large Cahier des charges}\\[2cm]
        
        % Project name + Overview
        {\huge \bfseries Serinity}\\[0.5cm]
        \textsc{\Large Application desktop/web de Psychothérapie et Développement Personnel}\\[3cm]
        
        \textsc{\Large Souba}\\[0.5cm]
        \today
        \vfill
    \end{center}
\end{titlepage}

%---------------------------------------
% Members page
%---------------------------------------
\newpage
\thispagestyle{empty}

\begin{center}
    \vspace*{\fill}

    {\Huge \textbf{Membres du groupe}}\\[1.5cm]

    {\Large
    Zouari Omar\\[0.4cm]
    Ghaith Ben Salah\\[0.4cm]
    Safa Jelassi\\[0.4cm]
    Malek Dahmen\\[0.4cm]
    Saif Dkhaili\\[0.4cm]
    Oussema Bouabdallah
    }

    \vspace*{\fill}
\end{center}
\newpage

%---------------------------------------
% Table of content(s)
%---------------------------------------
\renewcommand{\contentsname}{Sommaire}
\tableofcontents
\newpage

%---------------------------------------
\section{Thème du projet}
\textbf{Psychothérapie et développement personnel}

%---------------------------------------
\section{Problématique du projet}
L’accès au bien-être mental et au développement personnel reste limité pour une grande partie du public, qui manque d’outils fiables pour 
suivre et analyser ses émotions. Les psychothérapeutes, de leur côté, disposent rarement de solutions numériques performantes permettant 
un suivi structuré, une personnalisation des interventions et une analyse prédictive de l’évolution psychologique des patients. 
L’utilisation d’une application desktop/web intégrant l’intelligence artificielle peut combler ces lacunes, en offrant à la fois un accompagnement 
personnalisé pour les utilisateurs et un outil professionnel avancé pour les thérapeutes, tout en respectant la confidentialité et la sécurité des données.

%---------------------------------------
\section{Idée générale du projet}
Le projet vise à développer une application desktop/web intelligente dédiée à la psychothérapie et au développement personnel, 
destinée à la fois au grand public et aux psychothérapeutes. Elle intègre une couche d’IA pour améliorer le bien-être mental des utilisateurs 
et optimiser le suivi thérapeutique professionnel.

%---------------------------------------
\section{Idée détaillée du projet}
Ce projet consiste à concevoir une application desktop/web dédiée à la psychothérapie et au développement personnel, s’adressant à deux catégories d’utilisateurs:
\begin{itemize}
  \item \textbf{Grand public:} personnes non professionnelles souhaitant améliorer leur bien-être mental, mieux comprendre leurs émotions et développer 
    des stratégies personnelles de gestion du stress et de l’anxiété.
  \item \textbf{Psychothérapeutes / Psychologues:} utilisation de l’application comme un outil numérique professionnel pour accompagner les patients, 
    organiser les séances thérapeutiques et suivre l’évolution psychologique.
\end{itemize}

L’application intègre une couche d’intelligence artificielle capable d’analyser les données émotionnelles des utilisateurs 
(journaux personnels, résultats de tests psychologiques, comportements d’utilisation) et propose des recommandations personnalisées 
(exercices de relaxation, méditations guidées, plans de développement personnel).

Pour les professionnels, l’IA offre des fonctionnalités de suivi clinique, d’analyse prédictive, d’aide à la décision thérapeutique et 
de personnalisation des programmes de soins. L’ensemble du système est sécurisé, éthique et respectueux de la confidentialité des données.

%---------------------------------------
\section{Objectifs du projet}
\begin{itemize}
    \item Favoriser le bien-être mental et le développement personnel grâce à des recommandations intelligentes basées sur l’IA.
    \item Offrir un espace sécurisé d’expression et d’analyse des émotions.
    \item Aider les utilisateurs à gérer le stress, l’anxiété et leurs émotions de manière personnalisée.
    \item Fournir aux psychothérapeutes un outil intelligent de suivi et d’aide à la décision thérapeutique.
    \item Améliorer la collaboration patient–thérapeute tout en garantissant la confidentialité des données.
\end{itemize}

% --- ODD3 sub-section ---
\subsection*{Lien avec les Objectifs de Développement Durable (ODD)}
\subsubsection*{ODD 3 : Bonne santé et bien-être}

Le projet contribue directement à l’Objectif de Développement Durable 3, qui consiste à ``assurer une vie saine et promouvoir le bien-être de tous à tout âge''. 
Grâce à l’application, les utilisateurs peuvent suivre leur sommeil, leur humeur et accéder à des exercices de développement personnel et à des ressources thérapeutiques, tandis que les psychothérapeutes peuvent améliorer le suivi et l’efficacité de leurs consultations. 
L’intégration de l’intelligence artificielle permet de personnaliser les recommandations et d’optimiser le bien-être mental des utilisateurs.

\begin{figure}[!ht]
    \centering
    \includegraphics[width=0.4\textwidth]{../img/general/odd-3.png}
    \caption{ODD 3 : Bonne santé et bien-être}
    \label{fig:odd3}
\end{figure}
\newpage

%---------------------------------------
% `Étude de l’existant` section
%---------------------------------------
\section{Étude de l’existant}

\begin{longtable}{|p{5cm}|p{5cm}|p{5cm}|}
\hline
\textbf{Solution} & \textbf{Avantages} & \textbf{Limites} \\
\hline
\endfirsthead
\hline
\textbf{Solution} & \textbf{Avantages} & \textbf{Limites} \\
\hline
\endhead
Headspace, Calm, Moodfit & Exercices de méditation, suivi de l’humeur, conseils de bien-être & Recommandations génériques, pas de suivi personnalisé, 
  pas d’analyse prédictive \\
\hline
SimplePractice, Theranest & Gestion des patients, organisation des séances, stockage sécurisé des notes & Peu d’intelligence artificielle, 
  peu d’aide à la décision clinique, pas de suivi émotionnel automatique \\
\hline
Applications IA (prototypes) & Analyse de l’humeur, recommandations personnalisées & Souvent isolées, pas de solution complète combinant 
  suivi émotionnel, recommandations et outils professionnels \\
\hline
\end{longtable}

\textbf{Conclusion:} Il n’existe pas encore d’application desktop/web complète qui combine IA pour l’analyse émotionnelle et recommandations personnalisées, 
suivi professionnel pour les psychothérapeutes, et sécurité/confidentialité optimale, créant ainsi une opportunité pour ce projet.

%---------------------------------------
\section{Architecture g\'en\'erale du syst\`eme}

Le syst\`eme propos\'e repose sur une architecture modulaire orient\'ee services (SOA), permettant une s\'eparation claire des responsabilit\'es entre les
diff\'erents composants fonctionnels.  
Chaque module est con\c{c}u pour \^etre ind\'ependant tout en communiquant avec les autres via des interfaces bien d\'efinies.

L’architecture int\`egre \'egalement plusieurs composants d’intelligence artificielle afin d’am\'eliorer l’exp\'erience utilisateur, 
d’automatiser certaines t\^aches cliniques et d’optimiser la gestion des ressources.

%---------------------------------------
\section{Description des modules}

%--------------------------
\subsection{Module Gestion des Utilisateurs (Access-Control) - @ZouariOmar}

Ce module assure la gestion compl\`ete des utilisateurs de la plateforme, incluant l’authentification, l’autorisation, la gestion des r\^oles et la s\'ecurit\'e des acc\`es.  
Il constitue le point d’entr\'ee principal du syst\`eme.

\textbf{Entit\'es principales:}
\begin{itemize}
  \item \textbf{User}:
  repr\'esente un utilisateur de la plateforme (client, th\'erapeute ou administrateur), avec ses informations d’authentification et son \'etat de compte.

  \item \textbf{Profile}:
  contient les informations personnelles associ\'ees \`a un utilisateur.

  \item \textbf{Role}:
  d\'efinit les r\^oles du syst\`eme (CLIENT, THERAPISTE, ADMIN) et leurs responsabilit\'es.

  \item \textbf{Permission}:
  repr\'esente une autorisation sp\'ecifique accord\'ee \`a un r\^ole pour acc\'eder \`a certaines fonctionnalit\'es.

  \item \textbf{UserRole}:
  table d’association permettant d’attribuer un ou plusieurs r\^oles \`a un utilisateur.

  \item \textbf{AuthSession}:
  g\`ere les sessions d’authentification et assure la s\'ecurit\'e des connexions.

  \item \textbf{AuditLog}:
  permet la tra\c{c}abilit\'e des actions sensibles effectu\'ees par les utilisateurs.
\end{itemize}

%--------------------------
\subsection{Module Gestion du Sommeil - @SafaJelassi}

Ce module permet le suivi, l’analyse et l’optimisation des habitudes de sommeil des utilisateurs afin d’am\'eliorer leur bien-\^etre physique et mental.

\textbf{Entit\'es principales:}
\begin{itemize}
  \item \textbf{Cycle}:
  repr\'esente un cycle de sommeil enregistr\'e par l’utilisateur, incluant l’heure de d\'ebut, l’heure de fin, 
    la dur\'ee totale, la qualit\'e du sommeil et des remarques personnelles.

  \item \textbf{Dreams}:
  repr\'esente les r\^eves associ\'es \`a un cycle de sommeil, en pr\'ecisant les \'emotions ressenties, 
    l’impact psychologique, un score d’intensit\'e et les motifs r\'ecurrents.
\end{itemize}

%--------------------------
\subsection{Module Gestion de l’Humeur - @GhaithBenSalah}

Ce module permet aux utilisateurs de suivre leur \'etat \'emotionnel quotidien, d’identifier leurs \'emotions dominantes et de 
tenir un journal personnel afin de mieux comprendre leur bien-\^etre psychologique.

\textbf{Entit\'es principales:}
\begin{itemize}
  \item \textbf{Mood}:
  repr\'esente l’humeur quotidienne de l’utilisateur, incluant l’\'emotion ressentie, son intensit\'e et une description optionnelle.

  \item \textbf{Questions}:
  repr\'esente un ensemble de questions guid\'ees permettant d’analyser plus en profondeur l’\'etat \'emotionnel de l’utilisateur.

  \item \textbf{Choices}:
  repr\'esente les diff\'erentes options de r\'eponse propos\'ees pour chaque question.

  \item \textbf{Journal}:
  permet \`a l’utilisateur de r\'ediger des notes personnelles li\'ees \`a son humeur et \`a ses exp\'eriences quotidiennes.
\end{itemize}

%--------------------------
\subsection{Module R\'eseau de Soutien (Forum) - @SaifDkhaili}

Ce module offre un espace communautaire permettant aux utilisateurs d’\'echanger, de partager leurs exp\'eriences, de poser des questions et de se soutenir mutuellement dans un cadre s\'ecuris\'e.

\textbf{Entit\'es principales:}
\begin{itemize}
  \item \textbf{Post}:
  repr\'esente une publication cr\'e\'ee par un utilisateur pour partager une exp\'erience, poser une question ou initier une discussion.

  \item \textbf{Comment}:
  repr\'esente une r\'eponse ou une interaction d’un utilisateur sur un post afin d’apporter un soutien ou un avis.
\end{itemize}

%--------------------------
\subsection{Module Exercices et Ressources - @MalekDahmen}

Ce module propose des exercices guid\'es et des ressources multim\'edia visant \`a favoriser la relaxation, la gestion du stress et l’am\'elioration du bien-\^etre mental.

\textbf{Entit\'es principales:}
\begin{itemize}
  \item \textbf{Exercises}:  
  repr\'esente un exercice guid\'e tel que la respiration, la m\'editation ou la relaxation musculaire, avec son niveau et sa dur\'ee.

  \item \textbf{Resources}:  
  repr\'esente une ressource multim\'edia (audio, vid\'eo ou texte) mise \`a disposition des utilisateurs pour l’accompagnement et l’apprentissage.

  \item \textbf{ExerciseSession}:  
  repr\'esente une session de r\'ealisation d’un exercice par un utilisateur, incluant l’\'etat d’ach\`evement et un retour personnel.

  \item \textbf{Favorite}:  
  permet \`a l’utilisateur d’enregistrer ses exercices ou ressources pr\'ef\'er\'es pour un acc\`es rapide.
\end{itemize}

%--------------------------
\subsection{Module Prise de Rendez-vous et Consultation en Ligne - @OussemaBouabdallah}

Ce module permet la planification et la gestion des rendez-vous entre les clients et les th\'erapeutes, ainsi que les consultations en ligne.

\textbf{Entit\'es principales:}
\begin{itemize}
  \item \textbf{Appointment}: repr\'esente un rendez-vous planifi\'e.
  \item \textbf{Consultation}: session de consultation en ligne.
  \item \textbf{Availability}: g\`ere les disponibilit\'es des th\'erapeutes.
\end{itemize}

%---------------------------------------
\section{Mod\`ele de donn\'ees}

Le mod\`ele de donn\'ees est con\c{c}u autour de relations claires entre utilisateurs, clients, th\'erapeutes et leurs interactions.  
Il garantit la coh\'erence des donn\'ees tout en facilitant l’\'evolution future du syst\`eme.

Les relations principales sont:
\begin{itemize}
  \item Un utilisateur peut \^etre un client ou un th\'erapeute
  \item Un client peut avoir plusieurs rendez-vous, s\'eances et \'evaluations
  \item Un th\'erapeute peut g\'erer plusieurs disponibilit\'es et rendez-vous
\end{itemize}

%---------------------------------------
\section{Int\'egration de l’Intelligence Artificielle}

L’intelligence artificielle joue un r\^ole central dans le syst\`eme propos\'e et est int\'egr\'ee dans plusieurs modules 
pour am\'eliorer l’exp\'erience utilisateur et l’efficacit\'e des processus:

\begin{itemize}
  \item \textbf{Face Recognition:} authentifie les utilisateurs via reconnaissance faciale pour renforcer la s\'ecurit\'e des acc\`es.
  \item \textbf{Journal-NLP (KNN):} analyse les journaux clients pour identifier et taguer les \'emotions (anxi\'et\'e, tristesse, col\`ere, calme, motivation).
  \item \textbf{OptaPlanner:} optimise la planification des disponibilit\'es et l’affectation des rendez-vous pour les th\'erapeutes.
  \item \textbf{Session note summarization \& topic extraction:} r\'esume automatiquement les notes de s\'eance et en extrait les th\`emes principaux.
  \item \textbf{Auto-assessment:} analyse les r\'esultats des tests psychologiques (QCM ou images) et propose une interpr\'etation automatique.
  \item \textbf{Appointment recommendation:} propose le meilleur cr\'eneau de rendez-vous en analysant les disponibilit\'es, l’historique et des contraintes simples.
\end{itemize}

Cette int\'egration permet une prise en charge plus personnalis\'ee et efficace des clients, tout en r\'eduisant la charge administrative des th\'erapeutes.

%---------------------------------------
% Diagrams section
%---------------------------------------
\newpage
\section{Diagrammes de l'application}

\subsection{Diagramme de cas d'utilisation (Use Case)}

Le diagramme de cas d'utilisation illustre les interactions entre les différents acteurs et le système.

\begin{figure}[!ht]
    \centering
    \includegraphics[width=0.6\textwidth]{../img/diagram/use-case.png}
    \caption{Diagramme de cas d'utilisation de l'application}
    \label{fig:use-case}
\end{figure}
\newpage

\subsection{Diagramme de classes}

Le diagramme de classes montre la structure des classes principales et leurs relations dans l'application.

\begin{figure}[!ht]
    \centering
    \includegraphics[width=1\textwidth]{../img/diagram/class-diagram.png}
    \caption{Diagramme de classes de l'application}
    \label{fig:class-diagram}
\end{figure}

%---------------------------------------
% Charte Graphique et Identité Visuelle
%---------------------------------------
\section{Charte Graphique et Identité Visuelle}

Cette section pr\'esente les choix graphiques adopt\'es pour la plateforme \textit{Serinity}, visant \`a cr\'eer une interface apaisante, moderne et adapt\'ee au domaine du bien-\^etre et de la sant\'e mentale.

\subsection{Palette de Couleurs}

La palette de couleurs a \'et\'e s\'electionn\'ee afin de transmettre des valeurs de calme, de s\'ecurit\'e et de confiance, tout en assurant une bonne lisibilit\'e et une exp\'erience utilisateur agr\'eable.

\subsubsection{UI Color System}

\includegraphics[width=1\textwidth]{../img/general/color-palette.png}\\

\begin{itemize}
  \item \textbf{\#2F6F6D}: couleur principale, symbolisant le calme, l’\'equilibre et la s\'er\'enit\'e.
  \item \textbf{\#E6F2F1}: couleur de fond douce, utilis\'ee pour les sections et conteneurs.
  \item \textbf{\#88BDBC}: couleur secondaire, utilis\'ee pour les \'el\'ements interactifs.
  \item \textbf{\#1F2933}: couleur principale du texte, assurant un contraste \'elev\'e.
  \item \textbf{\#5F6C7B}: couleur de texte secondaire.
  \item \textbf{\#F9FAFB}: couleur de fond g\'en\'erale pour une interface claire et a\'er\'ee.
\end{itemize}

\subsubsection{Status Colors}

\begin{itemize}
  \item \textbf{\#4CAF50}: succ\`es, validation ou action positive.
  \item \textbf{\#F4B400}: avertissement ou information importante.
  \item \textbf{\#D9534F}: erreur ou action critique.
\end{itemize}

\subsection{Choix du Logo}

Le logo de la plateforme, intitul\'ee \textbf{Serinity}, repr\'esente une silhouette humaine en posture de m\'editation, entour\'ee de formes circulaires et de lignes symbolisant la connexion, l’harmonie et l’\'equilibre mental.

Ce choix visuel refl\`ete:
\begin{itemize}
  \item la s\'er\'enit\'e et le bien-\^etre psychologique,
  \item l’accompagnement th\'erapeutique,
  \item la dimension technologique de la plateforme \`a travers des motifs connect\'es.
\end{itemize}

Le logo s’int\`egre harmonieusement avec la palette de couleurs choisie, renfor\c{c}ant l’identit\'e visuelle et la coh\'erence graphique de la plateforme.

%---------------------------------------
% Maquettes / Screenshots
%---------------------------------------
\section{Maquettes de l'application}

\subsection{Version desktop}

\subsubsection{Interface de connexion}

\begin{figure}[!ht]
    \centering
    \includegraphics[width=0.8\textwidth]{../img/preview/desktop-login.png}
    \caption{Maquette de l'interface de connexion – Version desktop}
    \label{fig:desktop-login}
\end{figure}

\newpage
\subsubsection{Interface tableau de bord client}

\begin{figure}[!ht]
    \centering
    \includegraphics[width=0.9\textwidth]{../img/preview/desktop-dashboard.png}
    \caption{Maquette du tableau de bord client – Version desktop}
    \label{fig:desktop-dashboard}
\end{figure}

\newpage
\subsection{Version Web}

\subsubsection{Interface de connexion}

\begin{figure}[!ht]
    \centering
    \includegraphics[width=0.8\textwidth]{../img/preview/web-login.png}
    \caption{Maquette de l'interface de connexion – Version Web}
    \label{fig:web-login}
\end{figure}

\newpage
\subsubsection{Interface tableau de bord client}

\begin{figure}[!ht]
    \centering
    \includegraphics[width=0.9\textwidth]{../img/preview/web-dashboard.png}
    \caption{Maquette du tableau de bord client – Version Web}
    \label{fig:web-dashboard}
\end{figure}

%---------------------------------------
\newpage
\section{Conclusion}

Le syst\`eme propos\'e offre une solution compl\`ete et intelligente pour la gestion des services de sant\'e mentale.  
L’approche modulaire et l’int\'egration de l’intelligence artificielle assurent une plateforme \'evolutive, performante et centr\'ee sur l’utilisateur.

\end{document}
